\section{Completeness}
\begin{definition}
    A metric space is called complete if every Cauchy sequence converges.
\end{definition}
\begin{definition}
    A subset $A\subset M$ where $M$ is a metric space is called bounded if there is some $r>0$ and $z\in M$ such that $A\subset B_r(z)$.
\end{definition}
\begin{lemma}\label{ccb}
    Convergent $\implies$ Cauchy $\implies$ Bounded.
\end{lemma}
\begin{proof}
    Suppose $x_n\to x$, then $\forall\epsilon>0$, we can find some $N\in\mathbb N$ such that $\forall n>N,d(x,x_n)<\epsilon/2$, then $\forall n,m>N$,
    $$d(x_m,x_n)\le d(x_m,x)+d(x_n,x)<2\epsilon/2=\epsilon$$
    Assume that $(x_n)$ is Cauchy, we need that $\{x_n:n\in\mathbb N\}$ is contained in some ball.
    We know that there is some $N\in\mathbb N,\forall n,m>N, d(x_n,x_m)<\epsilon$.
    We can take $\epsilon=1$, so $\{x_n:n\in\mathbb N, n>N\}\subset B_1(x_N)$.
    Since $(x_n)_{n<N}$ is finite, it is contained in some ball $B$.
    In particular, we can take the ball $B_{\max\{1,d(x_N,x_i):i\le N\}}(x_N)$ contains the sequence.
\end{proof}
\begin{remark}
    Bounded does not imply Cauchy and Cauchy does not imply Convergent.
\end{remark}
\begin{definition}
    A metric space $M$ is complete if every Cauchy sequence converges.
\end{definition}
\begin{proposition}
    If $M,M'$ are complete, so is $M\oplus_pM'$.
\end{proposition}
\begin{proof}
    Let $(a_n)$ be Cauchy in the product, then say that $a_i=(x_i,x_i')$, then for all $m,n\in\mathbb N$, $\max\{d(x_m,x_n),d(x_m',x_n')\}\le d_p(a_m,a_n)$, so $(x_n),(x_n')$ are both Cauchy.
    Since both $M,M'$ are complete, $\exists x\in M,x'\in M', x_n\to x, x_n'\to x'$.
    Hence $a_n\to (x,x')$.
\end{proof}
\begin{example}
    $\mathbb R^n,\mathbb C^n$ are complete (in Euclidean metric) for any $n$.
\end{example}
There is another very important example:
\begin{theorem}
    Let $S$ be a non-empty set, then $\ell_\infty S$ is complete under the uniform metric.
\end{theorem}
\begin{proof}
    Theorem \ref{GP_UnifConv} shows that a uniformly Cauchy sequence of functions does converge to some scalar function on $S$.
    To see it is bounded, choose $n\in\mathbb N$ such that $d(f_n,f)<1$, then since there is some $C\ge 0$ such that $\sup|f_n|\le C$, we have $|f|\le |f-f_n|+|f_n|<C+1$ so it is bounded as well.
\end{proof}
\begin{proposition}
    Let $N\subset M$ be a metric subspace.\\
    1. If $N$ is complete, then $N$ is closed in $M$.\\
    2. If $N$ is closed and $M$ is complete, so is $N$.
\end{proposition}
So a metric subspace of a complete space is complete if and only if it is closed.
\begin{proof}
    1. If $N$ is complete, let $(x_n)$ be a sequence in $N$ such that $x_n\to x$ in $M$, but this means that $(x_n)$ is Cauchy by Lemma \ref{ccb}, therefore it is convergent in $N$ due to completeness, hence $x\in N$ due to uniqueness of limit in metric space.\\
    2. Choose any Cauchy sequence $(x_n)$ in $N$, we know that $(x_n)\to x$ for some $x\in M$ due to completeness of $M$, but since $N$ is closed, $x\in N$ as well, so $N$ is complete.
\end{proof}
\begin{theorem}
    Let $M$ be a metric space, then the space of bounded continuous scalar functions in $M$, $C_b(M)$ is complete in the uniform metric.
\end{theorem}
\begin{proof}
    $C_b(M)$ is a metric subspace of $\ell_\infty(M)$ which is complete.
    But uniform limit of continuous functions to a continuous function, so $C_b(M)$ is closed.\\
    To spell out the proof, fix $x\in M, \epsilon>0$, we can choose $N$ such that $D(f_n,f)<\epsilon/3$ where $D$ is the uniform metric.
    Fix any $n\ge N$, then since $f_n$ is continuous, $\exists \delta>0,d(x,y)<\epsilon\implies |f_n(x)-f_n(y)|<\epsilon/3$.
    Hence $d(x,y)<\delta\implies |f(x)-f(y)|\le |f(x)-f_n(x)|+|f(y)-f_n(y)|+|f_n(x)-f_n(y)|<3\epsilon/3=\epsilon$.
\end{proof}
Fix some $S\neq\varnothing$, a metric space $(N,d')$.
Let $\ell_\infty (S,N)$ be the space of bounded functions $S\to N$.
Then we can define the uniform metric on $\ell_\infty(S,N)$ defined by $D(f,g)=\sup_{x\in S}d'(f(x),g(x))$.\\
Now given a metric space $(M,d)$, let $C_b(M,N)$ be the set fo bounded continuous functions $M\to N$, then we have
\begin{theorem}
    Let $S,M,N$ be as above, assuming that $N$ is complete, then $\ell_\infty(S,N)$ is complete under uniform metric, and since $C_b(M,N)$ is closed in $\ell_\infty(M,N)$ hence complete.
\end{theorem}
\begin{proof}
    Analogous to the case where $M=\mathbb R$ or $\mathbb C$.
\end{proof}
\begin{example}
    1. For any closed and bounded interval $[a,b]\in\mathbb R$, then continuous functions on $[a,b]$ are the continuous and bounded functions on $[a,b]$ is complete under the uniform metric.
\end{example}
\begin{definition}
    A map $f:M\to M'$ is a contraction mapping if $f$ is $L$-Lipschitz with $L<1$.
\end{definition}
\begin{theorem}[Contraction Mapping Theorem, aka Banach Fixed Point Theorem]\label{banach}
    If $f$ is a contraction mapping in a nonempty complete metric space, then $f$ has an unique fixed point.
\end{theorem}
Note that it is important for the condition listed to be satisfied.
\begin{example}
    1. If we remove the completeness criterion, $f:\mathbb R\setminus\{0\}\to\mathbb R\setminus\{0\}$ defined by $f(x)=x/2$, then $f$ is a contraction but do not have fixed point.\\
    2. If we remove $L<1$, $f:\mathbb R\to\mathbb R$ by $f(x)=x+1$ is $1$-Lipschitz but do not have any fixed point.\\
    3. $f(x)=x+1/x,[1,\infty)$
\end{example}
\begin{proof}
    Fix $x_0\in M$, then define a sequence $x_n$ by $x_{n+1}=f(x_n)$, so $x_n=f^n(x_0)$.
    We shall show that this sequence is Cauchy.
    For $n\ge 2$, $d(x_n,x_{n-1})\le Ld(x_{n-1},x_{n-2})\le L^{n-1}d(x_1,x_0)$ inductively.
    For $m>n$, 
    \begin{align*}
        d(x_m,x_n)&\le d(x_n,x_{n+1})+\cdots+d(x_{m-1},x_m)\\
        &\le(L^{m-1}+L^{m-2}+\cdots+L^n)d(x_1,x_0)\\
        &\le\frac{L^n}{1-L}d(x_1,x_0)
    \end{align*}
    The last term, which only depends on the smaller term $n$, can be as small as we want when $n$ is large enough, so the sequence is Cauchy.\\
    Hence there is a limit $x$ of the sequence $x_n$ since $M$, but since $f$ is continuous, $f(x_n)\to f(x)$, but $f(x_n)=x_{n+1}$, so by uniqueness of limits, $f(x)=x$.\\
    Suppose $f(x)=x$ and $f(y)=y$, then if $x\neq y$, $|x-y|=|f(x)-f(y)|\le L|x-y|<|x-y|$ which is a contradiction.
\end{proof}
Note that $x_n\to x$ exponentially fast, so it can also be applied to numerical analysis to find an approximated solution of the fixed point.\\
An application of the contraction mapping theorem is to analyze the existence and uniqueness of the solution of an initial value problem.
\begin{example}
    The IVP $f^\prime(t)=f(t^2), f(0)=y_0$ on $C[0,1/2]$ is what we are interested in.
    Assume that $f$ has a solution, then immediately $f$ is continuously differentiable.
    By FTC,
    $$f(t)=f(0)+\int_0^tf(x^2)\,\mathrm dx$$
    Let $M=C[0,1/2]$, which is nonempty and complete, then consider the mapping $T:M\to M$ defined by
    $(Tg)(t)=y_0+\int_0^tg(x^2)\,\mathrm dx$
    $T$ is trivially well-defined since $x\in [0,1/2]\implies x^2\in[0,1/4]\subset[0,1/2]$ and that $g(x^2)$ is continuous in $x$. Also by FTC, $(Tg)^\prime=g$, so $Tg$ is continuously differentiable hence continuous.
    Now $f$ solves the IVP iff $f$ is a fixed point of $T$.
    Also we can check that $T$ is a contraction.
    Indeed, take $g,h\in\mathbb M$, then
    \begin{align*}
        |Tg(t)-Th(t)|&=\left|\int_0^tg(x^2)-h(x^2)\,\mathrm dx\right|\\
        &\le\int_0^t|g(x^2)-h(x^2)|\,\mathrm dx\le tD(g,h)\le D(g,h)/2
    \end{align*}
    So it is a contraction mapping, hence by Theorem \ref{banach} a unique fixed point exists.
\end{example}
\begin{theorem}[Lindelof-Picard Theorem]
    Let $a<b, R>0$ be real numbers and $y-0=\mathbb R^n$.
    Suppose there is a continuous $\phi:[a,b]\times B_R(y_0)\to \mathbb R^n$.
    Assume $K>0$ such that $\forall x,y\in B_R(y_0),\forall t\in [a,b],\|\phi(t,x)-\phi(t,y)\|\le K\|x-y\|$, then $\exists\epsilon>0,\forall t_0\in [a,b]$, the IVP
    $$f^\prime(t)=\phi(t,f(t)),f(t_0)=y_0$$
    has a unique solution on $[t_0-\epsilon,t_0+\epsilon]\cap [a,b]$.
\end{theorem}
\begin{proof}
    Observe that $\mathbb R^n$ is complete, so $B_R(y_0)$ is closed hence complete, then since $\phi$ is continuous, it is bounded in the closed bounded set $[a,b]\times B_R(y_0)\subset\mathbb R^{n+1}$.
    Let
    $$C=\sup\{|\phi(t,x)|:t\in [a,b],x\in B_R(y_0)\},\epsilon=\min\{R/C,1/(2K)\}$$
    We want to solve the IVP on $[c,d]=[t_0-\epsilon,t_0+\epsilon]\cap[a,b]$.
    Now the set of functions
    $$M=C([c,d],B_R(y_0))$$
    is nonempty and complete.\\
    Consider the mapping $T:M\to M$ by
    $$(Tg)(t)=y_0+\int_0^t\phi(x, g(x))\,\mathrm dx$$
    Now $Tg$ is continuous for continuous $g$ by FTC (in fact $Tg$ is even continuously differentiable).
    Also $(Tg)^\prime(t)=\phi(t,g(t))$.
    In addition, $Tg$ takes values in $B_R(y_0)$ since
    $$\|(Tg)(t)-y_0\|=\left\|\int_{t_0}^t\phi(x, g(x))\,\mathrm dx\right\|\le\int_{t_0}^t\|\phi(x, g(x))\|\,\mathrm dx\le \epsilon C\le R$$
    It remains to show that it is a contraction mapping, then the theorem can be deduced from the contraction mapping theorem since the fixed point would be continuously differentiable and solves the differential equation.
    Also every solution is a fixed point.\\
    For $g,h\in\mathbb M$, then we consider the uniform distance of $Tg,Th$.
    Indeed,
    \begin{align*}
        \|Tg(t)-Th(t)\|&=\left\|\int_{t_0}^t\phi(s,g(s))-\phi(s,h(s))\,\mathrm ds\right\|\\
        &\le\int_{t_0}^t\|\phi(s,g(s))-\phi(s,h(s))\|\,\mathrm ds\\
        &\le\epsilon KD(g,h)\le D(g,h)/2
    \end{align*}
    for any $t\in\mathbb R$ by the Lipschitz condition we assumed.
    Therefore $T$ is a contraction mapping, and the proof is complete.
\end{proof}
\begin{remark}
    1. In general, however, you cannot extend the solution guaranteed above to a global solution.
    But in our previous example we can extend the solution to $[0,1)$.\\
    2. Also, we can apply the theorem to solve higher order equations by considering the vector of derivatives.\\
    3. If $f:[a,b]\to\mathbb R^n$ can be written as $(f_1,f_2,\ldots,f_n)$, so $f^\prime=(f_1^\prime,f_2^\prime,\ldots f_n^\prime)$ assuming each component is differentiable.
    Similarly, the integral of a vector valued function is the vector of the integrals of the components, given that they exist.
    So we can do everything by components.\\
    4. The reason that the integral of the norm is at least the norm of the integral is Cauchy-Schwarz.\\
    5. We can show that a continuous function on a closed bounded set in $\mathbb R^n$ is bounded by Bolzano-Weierstrass.
\end{remark}