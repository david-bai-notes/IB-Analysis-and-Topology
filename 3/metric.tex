\section{Metric Space}
\begin{definition}
    Let $M$ be an arbitrary set, a metric on $M$ is a function $d:M\times M\to\mathbb R_{\ge 0}$ such that the following properties hold:\\
    1. $\forall x,y\in M,d(x,y)=0\iff x=y$.\\
    2. $\forall x,y\in M,d(x,y)=d(y,x)$.\\
    3. $\forall x,y,z\in M, d(x,y)+d(y,z)\ge d(z,x)$.\\
    The couple $(M,d)$ is called a metric space.
\end{definition}
\begin{example}
    1. Let $M$ be $\mathbb R$ or $\mathbb C$, and $d(x,y)=|x-y|$ is called the usual metric on those sets.\\
    2. Take $M=\mathbb R^n$ or $\mathbb C^n$, then
    $$d((x_1,\ldots,x_n),(y_1,\ldots,y_n))=\sqrt{|x_1-y_1|^2+\ldots+|x_n-y_n|^2}$$.
    This is called the $\ell^2$ metric.\\
    3. Take the same $M$, then we can also have the $\ell^1$ metric where
    $$d((x_1,\ldots,x_n),(y_1,\ldots,y_n))=|x_1-y_1|+\ldots+|x_n-y_n|$$
    4. Also the same $M$, we have the $\ell^\infty$ metric where
    $$d((x_1,x_2,\ldots,x_n),(y_1,y_2,\ldots,y_n))=\max_{i}|x_i-y_i|$$
    5. We can have the $\ell^p$ metric for $p\ge 1$ where
    $$d((x_1,\ldots,x_n),(y_1,\ldots,y_n))=\sqrt[p]{|x_1-y_1|^p+\ldots+|x_n-y_n|^p}$$
    6. Let $S$ be a set and $M=\ell_\infty S$ be the set of bounded scalar function on $S$.
    The metric we can take is the uniform metric $d(f,g)=\sup_S|f-g|$, which is well-defined since $f,g$ are bounded.\\
    7. Let $M$ be any set, then define
    $$d(x,y)=\begin{cases}
        1\text{, if $x=y$}\\
        0\text{, otherwise}
    \end{cases}$$
    This is called the \textit{discrete metric} and the space $(M,d)$ the discrete metric space.\\
    8. Let $G$ be a group that is generated by a symmetric set $S$.
    Define $d(x,y)$ be the least integer $n\ge 0$ such that $n$ is the least number of generators to get from $X$ to $y$.\\
    This develops to the discipline called geometric group theory.\\
    9. Suppose $G$ is a connected (finite) graph, then we can define the distance between two vertices $x,y$ to be the length of the shortest path from $x$ to $y$.\\
    10. Riemannian metric in geometry.\\
    11. Take $M$ to be the integers and we fix a prime $p$.
    We can define the $p$-adic metric $d_p(x,y)$ to be $0$ if $x=y$ and $\|x-y\|_p=p^{-n}$ where $n$ is the greatest power of $p$ in the prime factorisation of $|x-y|$.
    It is obvious that it is a metric.
    The metric space $(\mathbb Z,d_p)$ is called the $p$-adic integers.\\
    12. Let $M$ be the set of all functions from $\mathbb N$ to $\mathbb R$, that is, the set of all sequences.
    So for $x=(x_n),y=(y_n)$, we define $d(x,y)=\sum_{n=0}^\infty 2^{-n}\min\{1,|x_n-y_n|\}$.
\end{example}
As in before, we can construct new objects from old.
\begin{definition}
    Let $(M,d)$ be a metric space and $N\subseteq M$, then $(N,d|_{N\times N})$ is a metric space and is called the metric subspace of $(M,d)$.
    Sometimes we denote $(N,d|_{N\times N})$ by $(N,d)$.
\end{definition}
\begin{example}
    $C[0,1]$ be the set of real continuous functions on the unit interval having the uniform metric is a subspace of $l_\infty[0,1]$.
\end{example}
Note that there are other metrics on $C[0,1]$, for example
$$d(f,g)=\int_0^1|f(x)-g(x)|\,\mathrm dx$$
We also have the $L^2$ metric
$$d(f,g)=\sqrt{\int_0^1|f(x)-g(x)|\,\mathrm dx}$$
Note that the $L^\infty$ metric is the uniform metric.
\begin{definition}
    Let $(M,d),(N,d')$ be two metric spaces, we can define the metric product space by taking the underlying set $M\times N$ and the metric
    $$d_p((m_1,n_1),(m_2,n_2))=(d(m_1,m_2)^p+d'(n_1,n_2)^p)^{1/p}$$
    for some $p\ge 1$ or
    $$d_\infty((m_1,n_1),(m_2,n_2))=\max(\{d(m_1,m_2),d'(n_1,n_2)\})$$
    We can generalize it to any finite product of metric space.
\end{definition}
We need to generalize to topological spaces to have the notion of a quotient space in a metric/topological space.\\
We can introduce (uniform) convergence again in any metric spaces.
We work in a metric space $(M,d)$.
\footnote{When $d$ is understood, we do not usually state explicitly our metric $d$}
\begin{definition}
    Given a sequence $x_n$ in $M$ and a point $x\in M$, we say $x_n\to x$ as $n\to\infty$ if for any $\epsilon>0,\exists N\in\mathbb N,\forall n>N,d(x_n,x)<\epsilon$.\\
    Conversely, if such an $x$ exists for a sequence $x_n$, we say $x_n$ is convergent.
\end{definition}
\begin{lemma}
    Assume $x_n\to x$ and $x_n\to y$, then $x=y$.
\end{lemma}
\begin{proof}
    Assume for the sake of contradiction that it is not the case.
    We let $\epsilon=d(x,y)$, we have a large enough $N\in\mathbb N$ such that $n\in\mathbb N\implies d(x_n,x)<\epsilon/2, d(x_n,y)<\epsilon/2$, so
    $$\epsilon=d(x,y)\le d(x,x_n)+d(y,y_n)<2\epsilon/2=\epsilon$$
    A contradiction.
\end{proof}
Now we can introduce the concept of limit.
\begin{definition}
    Let $(x_n)$ be a convergent sequence which converges to $x$, we write
    $$\lim_{n\to\infty}x_n=x$$
\end{definition}
\begin{example}
    1. In $\mathbb R, \mathbb C$, this is the usual notion of convergence.\\
    2. Take the integers under the $2$-adic metric, $2^n\to 0$ as $n\to\infty$.\\
    3. A sequence which is eventually constant converges to that constant.
    Obviously the converse is false in general, but true in discrete metric spaces.\\
    4. Choose a nonempty set $S$, then functions that converge under the (induced) uniform metric on $\ell_\infty S$ converges uniformly as functions.
    This sometimes can work even on functions $\notin\ell_\infty S$, for example take $S=\mathbb R$, then $f_n(x)=x+1/n$ converges uniformly to the identity function.\\
    5. Consider the space $\mathbb R^{\mathbb N}$, the set of all sequences on $\mathbb R$ with the metric
    $$d((x_n),(y_n))=\sum_{n=1}^\infty 2^{-n}\min{\{1, |x_n-y_n|\}}$$
    then we can show that a sequence of sequences $(x^{(n)}_k)$ (where each $n$ gives a sequence) converges to a sequence $(x_k)$ if and only if for each $i$, $x^(n)_i\to x_i$.\\
    In fact, if we fix a set $S$, is there always a metric $d$ on $\mathbb R^S$ such that $f_n\to f$ under the metric $d$ if $f_n\to f$ pointwise on $S$?
    The answer is no, as we will need topological tools for it.\\
    6. Consider $C[0,1]$ under the uniform metric.
    Surely the function defined by $f_n=x^n$ do not converge, but if we equip $C[0,1]$ with a different metric, for example the $L^1$ metric, that is,
    $$d(f,g)=\int_0^1|f(x)-g(x)|\,\mathrm dx$$
    In this case, the sequence $f_n$ does converge to $0$.\\
    7. Let $(M,d),(M',d')$ be two metric spaces, we consider the metric product space $M\oplus_pM'=(M\times M',d_p)$.
    Then $(x_n,y_n)\to (x,y)$ if and only if $x_n\to x,y_n\to y$.\\
    8. Consider the metric subspace $N\subset M$.
    If a sequence $x_n$ converge to $x$ in $N$, then $x_n\to x$ in $M$.
    The converse is not true since we can take $N=M\setminus\{x\}$ if $x_n\to x$.
\end{example}
\begin{definition}
    Let $(M,d),(M',d')$ be two metric spaces and $f:M\to M'$ be a function.
    We say $f$ is continuous at $a\in M$ if
    $$\forall\epsilon>0,\exists\delta>0,\forall b\in M,d(a,b)<\delta\implies d'(f(a),f(b))<\epsilon$$
    If $f$ is continuous at every $a\in N\subseteq M$, then we say $f$ is continuous on $N$.
\end{definition}
Note that if $f$ is continuous on $M$, it is continuous on any $N\subseteq M$.
The converse, however, is not true.
\begin{example}
    We can take both metric spaces to be $\mathbb R$, then consider the function
    $$f(x)=\begin{cases}
        1\text{, if $x\neq 0$}\\
        0\text{, otherwise}
    \end{cases}$$
    Then $f$ is continuous on $N=\mathbb R\setminus\{0\}$ but not on $M=\mathbb R$.
\end{example}
\begin{proposition}
    $f$ is continuous at $a$ if any only if for any sequence $x_n\to a$, we have $f(x_n)\to f(a)$.
\end{proposition}
\begin{proof}
    If $f$ is continuous at $a$, then $\forall\epsilon>0$, we can find some $\delta>0$ such that $d(a,b)<\delta\implies d'(f(a),f(b))<\epsilon$.
    Now, find any $x_n\to a$, we can find $N\in\mathbb N, \forall n>N, d(a,x_n)<\delta$, but with the same $N$, $\forall n>N$, we have $d'(f(a),f(x_n))<\epsilon$ by the above.
    So $f(x_n)\to f(a)$.\\
    Conversely, if $x_n\to a\implies f(x_n)\to f(a)$ but $f$ is not continuous at $a$, then we can find $\epsilon>0$ such that $\forall\delta>0$, there is some $x\in M$ such that $d(x,a)<\epsilon$ but $d'(f(x),f(a))>\epsilon$.
    We may set $\delta_n=1/n$ and we can obtain the corresponding $x_n$.
    Now $x_n\to a$ but $f(x_n)\not\to f(a)$.
    This is a contradiction.
\end{proof}
The following two corollaries are then obvious.
\begin{corollary}
    Let $f$ and $g$ be continuous scalar functions, then $f+g, f\times g$ and $f/g$ (providing that $\forall x,g(x)\neq 0$) are all continuous.
\end{corollary}
\begin{corollary}
    If $f:M\to M',g:M'\to M''$ are both continuous, then $g\circ f$ is continuous.
\end{corollary}
One can also prove them using $\epsilon-\delta$, which is not hard either.
\begin{example}
    1. Constant, identity (equipping the same metric) and inclusion (in the sense of metric subspace) functions are continuous.\\
    2. Real and complex polynomials are continuous.\\
    3. The metric function itself is continuous (in fact Lipschitz) with respect to the $d_p$ metric on $M\times M$.
\end{example}
\begin{definition}
    A function $(M,d)\to (M',d')$ is Lipschitz continuous if there is some $C\ge 0$ such that
    $$\forall x,y\in M,d'(f(x),f(y))\le Cd(x,y)$$
    we sometimes call $f$ to be $C$-Lipschitz.
\end{definition}
\begin{proposition}
    A Lipschitz function is uniformly continuous.
\end{proposition}
\begin{proof}
    Trivial.
\end{proof}
\begin{definition}
    A map $g:(N,d)\to (N',d')$ is isometric if
    $$\forall x,y\in N,d'(g(x),g(y))=d(x,y)$$
    Note that an isometric function is $1$-Lipschitz.
    It also implies injective.
\end{definition}
We continue with examples.
\begin{example}
    4. Let $M,M'$ be metric spaces, fixing $y\in M'$, $f:M\to M\oplus_pM'$ by $f(x)=(x,y')$ is isometric, hence also ($1$-)Lipschitz.\\
    5. Let $(M,d), (M',d')$ be metric spaces.
    Consider $q:M\oplus_pM'\to M, q':M\oplus_pM'\to M'$ be the projection functions.
    Both of these functions are $1$-Lipschitz. %sometimes it is also isometric like Rn
    We can easily extend it to a finite product of metric spaces. 
\end{example}
Now we go on to talk about the topology of metric spaces.
We start with two observations.
Firstly, in a product metric space $M\oplus_pM'$, convergence does not depend on the value of $p$.
Secondly, continuity depends on the convergent sequences.
\begin{definition}
    We fix a metric space $(M,d)$, for $x\in M$ and $r\ge 0$, the open ball $D_r(x)$ is the set $\{y\in M:d(x,y)<r\}$.
\end{definition}
So $x_n\to x$ if and only if $\forall\epsilon>0,\exists N\in\mathbb N, n>N\implies x_n\in D_\epsilon(x)$.
And $f:M\to M'$ is continuous at $a\in M$ if and only if $\forall\epsilon>0,\exists\delta>0,\forall x\in M,x\in D_\delta(a)\implies f(x)\in D_\epsilon(f(a))$.
\begin{definition}
    On $(M,d)$, for $x\in M$ and $r\ge 0$, the closed ball $B_r(x)$ is the set $\{y\in M:d(x,y)\le r\}$.
\end{definition}
\begin{example}
    1. When $M$ is the real numbers, then an open ball is an open interval and closed ball is an closed interval.\\
    2. In $\mathbb R^2$, $B_1(0,0)$ is the unit disk with boundary in $d_2$, and an slanted square in $d_1$, and a big square in $d_\infty$.\\
    3. If $M$ is discrete, $D_1(x)=\{x\},B_1(x)=M$.
\end{example}
Note that $B_s(x)\subset D_r(x)\subset B_r(x)$ for any $s<r$.
\begin{definition}
    A subset $U\subset M$ with $x\in U$ is called a neighbourhood of $x$ (in $M$) if there exists some $r>0$ with $D_r(x)\subset U$.
\end{definition}
It does not matter if we take the closed ball instead.
\begin{definition}
    Given $U\subset M$, we say $U$ is open if $\forall x\in U, \exists r>0, D_r(x)\subset U$.
\end{definition}
So $U$ is open if and only if $U$ is a neightbourhood of $x$ for any $x\in U$.
\begin{lemma}
    Open balls are open.
\end{lemma}
\begin{proof}
    Immediate from definition but let us write the proof anyways.\\
    Consider $D_r(x)$, then for any $y\in D_r(x)$, since $d(x,y)<r$, if $z\in D_{r-d(x,y)}(y)$, then $d(x,z)\le d(y,z)+d(x,y)<r\implies D_{r-d(x,y)}(y)\subset D_r(x)$.
\end{proof}
\begin{proposition}\label{metric_nbhdconv}
    In a metric space $M$, the followings are equivalent:\\
    1. $x_n\to x$.\\
    2. For any neighbourhood $U$ of $x$, there is some $N\in\mathbb N, \forall n>N, x_n\in U$.\\
    3. For any open set $U$ containing $x$, there is some $N\in\mathbb N, \forall n>N, x_n\in U$.
\end{proposition}
\begin{proof}
    $1\implies 2$: $\exists r>0,D_r(x)\subset U$, so we can choose an $N\in\mathbb N,\forall n>N, d(x_n,x)<r\implies x_n\in U$.\\
    $2\implies 3$: Immediate by the preceding lemma.\\
    $3\implies 1$: Given $\epsilon>0$, take $U=D_\epsilon(x)$, then two statement becomes identical.
\end{proof}
\begin{proposition}\label{metric_preimage}
    Given function $f:M\to M'$, then:\\
    (A) For $a\in M$, the followings are equivalent:\\
    1. $f$ continuous at $a$.\\
    2. For any neighbourhood $V$ of $f(a)$, there is a neighbourhood $U$ of $a$ such that $f(U)\subset V$.\\
    3. For any neighbourhood $V$ of $f(a)$, $f^{-1}(V)$ is a neighbourhood of $a$.\\
    (B) The followings are equivalent:\\
    1. $f$ is continuous.\\
    2. The pre-image of any open set is open.
\end{proposition}
\begin{proof}
    Part (A):\\
    $1\implies 2$: Given any neighbourhood $V$ of $f(a)$, there is some $r$ such that $D_r(f(a))\subset V$.
    Since $f$ is continuous at $a$, there is $\delta>0$ with $f(D_\delta(a))\subset D_r(f(a))\subset V$.
    So $U=D_\delta(a)$ works.\\
    $2\implies 3$: Trivial since there is some neighbourhood $U$ containing $a$ with $f(U)\subset V\implies U\subset f^{-1}(V)$ so it is a neighbourhood of $a$.\\
    $3\implies 1$: Given $\epsilon>0$, $f^{-1}(D_\epsilon(f(a)))$ contains some open ball $D_\delta(a)$ for some $\delta>0$, so it's done.\\
    Part (B):\\
    $1\implies 2$: Given $V$ open in $M'$, for $x\in f^{-1}(V)$, we have $f(x)\in V$, so $V$ is a neighbourhood of $f(x)$.
    Since $f$ is continuous, by (A), there is an neighbourhood of $x$ containing in it.\\
    $2\implies 1$: We shall show that it is continuous at every point.
    Given $\epsilon>0,a\in M$, the ball $D_\epsilon(f(a))$ is open in $M'$, so $f^{-1}(D_\epsilon(f(a)))$ is open, so there is some $\delta$ with $D_\delta(a)\subset f^{-1}(D_\epsilon(f(a)))$, so we are done.
\end{proof}
\begin{definition}
    The topology of a metric space the collection of open subsets of it.
\end{definition}
\begin{proposition}\label{metric_topology}
    In a metric space $M$, we have the following:\\
    1. $\varnothing,M$ are open.\\
    2. If $\{U_i\}_{i\in I}$ are open, then $\bigcup_{i\in I}U_i$ is open.
    3. If $U,V\subset M$ are open, then $U\cap V$ is open.
\end{proposition}
\begin{proof}
    1 is trivial.\\
    For 2, given $x$ in the union, then there is a $j\in I$ such that $x\in U_j$, but then there is some open ball $U\ni x$ such that $U\subset U_j$, then $U$ is a subset of that union.
    So this union is open.\\
    Regarding 3, given $x\in U\cap V$, then there are $\epsilon_U,\epsilon_V>0$ such that $D_{\epsilon_U}(x)\subset U,D_{\epsilon_V}(x)\subset V$, so the ball $D_{\min\{\epsilon_U,\epsilon_V\}}(x)\subset U\cap V$.
\end{proof}
\begin{definition}
    A subset $A\subset M$ is closed if whenever $x_n\to x$ in $M$ for some sequece $(x_n)\in A$, then $x\in A$.
\end{definition}
\begin{example}
    1. Closed balls are closed.\\
    2. So in $\mathbb R$, any closed interval is closed.
    Also $\mathbb R$ itself is both open and closed.
    $[0,1)$ is neither open nor closed.
\end{example}
\begin{lemma}\label{metric_complement}
    A subset $A\subset M$ is closed if and only if $M\setminus A$ is open.
\end{lemma}
\begin{proof}
    If $A$ is closed but $M\setminus A$ is not open, so there is some $x\in M\setminus A$ such that $D_r(x)$ is not contained in $M\setminus A$ for all $r>0$.
    Hence for each $\epsilon>0, \exists x_\epsilon \in M, x_\epsilon\in D_\epsilon(x)$.
    Taking $a_n=x_{1/n}$ gives a contradiction.\\
    If $A$ is not closed but $M\setminus A$ is open.
    So we can find $(a_n)\in A$ such that $a_n\to a\notin A$, so there is some $\epsilon>0$ such that $D_\epsilon(a)\subset M\setminus A$, but this is a contradiction since $a_n\notin M\setminus A$ for any $n$ but it can go $\epsilon$-close to $a$.
\end{proof}
\begin{example}
    If $(N,d)$ is discrete, then every subset of $N$ is both open and closed.
\end{example}
\begin{definition}
    Two metrics on a set are equivalent if they give the same topology.
\end{definition}
Note that it is equivalent to say that the teo metrics have the same convergence sequences since they help identify the closed sets.
It also means that they have the same continuous functions, both to and from, any other spaces.
Note that the two metrics induce the same topology if and only if the identity maps from both spaces are continuous.
\begin{definition}
    A map $g:M\to M'$ is called a homeomorphism if it is a bijection and both $g$ and $g^{-1}$ are continuous.\\
    We say $g$ is an isometry if it is bijective and it is isometric.\\
    We say $M$ and $M'$ are homeomorphic if there is a homeomorphism between them.
    And $M$, $M'$ be isometric if there is an isometry between them.
\end{definition}
\begin{remark}
    1. Continuous bijections may not be a homeomorphism.
    Take $g:\mathbb R\to\mathbb R$ where the domain is equipped with discrete metric and the codomain with the usual metric.\\
    2. A surjective isometric function is an isometry.
\end{remark}
\begin{example}
    1. $(0,1),(0,\infty)$ are homeomorphic.
    Take $x\mapsto 1/x$.\\
    2. $\mathbb R^2$ and $\mathbb C$ are isometric. 
\end{example}
\begin{definition}
    Two metrics $d,d'$ on $M$ are uniformly equivalent if and only if both the identity functions $\operatorname{id}:(M,d)\to(M,d'),\operatorname{id}:(M,d')\to(M,d)$ are uniformly continuous.\\
    We say $d,d'$ are Lipschitz equivalent if and only if both the identity functions are Lipschitz.
\end{definition}
\begin{example}
    1. On $M\times M'$, $d_1,d_2,d_\infty$ are Lipschitz equivalent.\\
    2. (non-example) On $C[0,1]$ the uniform metric is not equivalent to the $L^1$ metric since they do not have the same convergent sequences.
\end{example}
