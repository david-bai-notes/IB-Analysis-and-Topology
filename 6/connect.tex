\section{Connectedness}
An interval $I$ in $\mathbb R$ has the defining property that $\forall x,y,z,x<y<z$, then $x,z\in I\implies y\in I$.
We know that a real continuous function maps intervals to intervals due to the intermediate value theorem.
But it may not work if the (restricted) domain is not an interval.
\begin{definition}
    A topological space $X$ is disconnected if there are open $U,V\subset X$ such that $U\neq\varnothing$ and $V\neq\varnothing$ partitions $X$, that is $U\cap V=\varnothing$ and $U\cup V=X$.
    In this case, we say $U,V$ disconnect $x$.\\
    A topological space $X$ is connected if it is not disconnected.
\end{definition}
\begin{lemma}\label{image_connected}
    The image of continuous function on connected space is connected.
\end{lemma}
\begin{proof}
    Suppose $f:X\to Y$ is continuous.
    Note that if we consider $f$ as $f:X\to\operatorname{Im}f$ then it is still continuous.
    Then if $U,V$ disconnect $\operatorname{Im}f$, then $f^{-1}(U),f^{-1}(V)$ disconnect $X$.
\end{proof}
\begin{theorem}\label{connected_eqv}
    For a topological space $X$, the followings are equivalent:\\
    1. $X$ is connected.\\
    2. If $f:X\to\mathbb R$ is continuous, then $f(X)$ is an interval.\\
    3. Every continous function $f:X\to D$, where $D$ is discrete and $|D|\ge 2$, is constant.
    \footnote{Most of the time we take $D=\mathbb Z$}
\end{theorem}
\begin{proof}
    $1\implies 2$: Obvious due to the preceding lemma and the trivial fact that an open set in $\mathbb R$ is connected if and only if it is an interval.\\
    $2\implies 3$: Immediate, also from the preceding lemma.\\
    $3\implies 1$: We shall prove the contrapositive.
    Suppose that $U,V$ disconnects $X$, then choose $d,e\in D$ with $d\neq e$, then the function $f$ defined by
    $$f(x)=
    \begin{cases}
        d\text{, if $x\in U$}\\
        e\text{, otherwise, that is if $x\in V$}
    \end{cases}$$
    is continuous but is not constant, contradiction.
\end{proof}
\begin{example}
    1. $\varnothing$ and singletons are connected.\\
    2. Any indiscrete topological space is connected.\\
    3. The cofinite topology on an infinite set is connected.\\
    4. The discrete topology is disconnected if it is not a singleton.
\end{example}
\begin{lemma}
    A subspace $Y\subset X$ is disconnected if and only if there are open sets $U,V\in X$ such that $U\cap Y\neq\varnothing, V\cap Y\neq\varnothing, U\cap V\cap Y=\varnothing, Y\subset U\cup V$.
\end{lemma}
\begin{proof}
    Trivial.
\end{proof}
\begin{proposition}\label{closure_connected}
    Let $Y$ be a connected subspace of $X$, then $\bar Y$ is connected.
\end{proposition}
\begin{proof}
    Assume not, then by the preceding lemma, there exists open sets $U,V$ in $X$ such that $U\cap\bar Y\neq\varnothing, V\cap\bar Y\neq\varnothing, U\cap V\cap\bar Y=\varnothing, \bar Y\subset U\cup V$.
    It follows that $U\cap V\cap Y=\varnothing, Y\subset U\cup V$, so we must have, WLOG, $U\cap Y=\varnothing$, then $Y\subset X\setminus U\implies\bar Y\subset X\setminus U\implies \bar Y\cap U=\varnothing$, contradiction.
\end{proof}
\begin{remark}
    1. Alternatively, we can use the third part of Theorem \ref{connected_eqv}.
    2. In fact, for any $Z$ with $Y\subset Z\subset\bar Y$ is connected since the closure of $Z$ is $\bar Y$.
\end{remark}
\begin{proof}[Alternative proof of Lemma \ref{image_connected}]
    Let $f:X\to Y$ be continuous, for convenience we can just assume $f$ is surjective using the same argument as the original proof, then consider any continuous $g:Y\to\mathbb Z$, then $g\circ f$ is continuous hence constant since $f$ is connected, but $f$ is surjective, so $g$ is constant, then it is done by Theorem \ref{connected_eqv}.
\end{proof}
\begin{remark}
    1. Connectedness is a topological property.\\
    2. If $f:X\to Y$ is continuous and $A\subset X$ and $A$ is connected, then $f(A)$ is connected.
\end{remark}
\begin{corollary}
    If $X$ is connected and $R$ an equivalence relation on $X$, then $X/R$ is connected.
\end{corollary}
\begin{proof}
    The quotient map is continuous and surjective.
\end{proof}
\begin{example}
    let $Y=\{(x,\sin(1/x)):x>0\}\subset\mathbb R^2$ is connected since it is the image of $f(x)=(x,\sin(1/x))$, which is continuous since its components are connected, over $\mathbb R_{>0}$.\\
    By Proposition \ref{closure_connected}, $\bar Y=Y\cup(\{0\}\times [-1,1])$ is also connected.
    This is called the Topologist's Sine Wave.
\end{example}
\begin{lemma}\label{union_connected}
    Let $\mathscr A$ be a family of connected subset of a topological space $X$ such that $\forall A,B\in\mathscr A,A\cap B=\varnothing$, then $\bigcup_{A\in\mathscr A}A$ is connected.
\end{lemma}
\begin{proof}
    Suppose $f:\bigcup_{A\in\mathscr A}A\to\mathbb Z$ is connected, then $f|_A$ is continuous for any $A\in\mathscr A$, thus it is constant, say it is $n_A$, then $\forall A,B\in\mathscr A$, then $n_A=n_B$ since $A\cap B\neq\varnothing$.
    Thus $f$ is constant, hence $\bigcup_{A\in\mathscr A}A$ is connected.
\end{proof}
\begin{proposition}
    If $X,Y$ are connected, so is $X\times Y$.
\end{proposition}
\begin{proof}
    Observe that $\forall x\in X,\{x\}\times Y\cong Y$ is connected and $\forall y\in Y, X\times \{y\}\cong X$ is connected as well, so since $(x,y)\in (\{x\}\times Y)\cap(X\times \{y\})\neq\varnothing$, by the preceding lemma $A_{x,y}=(\{x\}\times Y)\cup(X\times \{y\})$ is connected.
    Now obviously $(x,y')\in A_{x,y}\cap A_{x',y'}\neq\varnothing$, so $X\times Y=\bigcup_{x\in X,y\in Y}A_{x,y}$ is connected by the preceding lemma.
\end{proof}
\begin{definition}
    Let $X$ be a topological space, we define an equivalence relation $R$ by $xRy$ if and only if there is a connected $U\subset X$ such that $x,y\in U$.
    One can check that this is an equivalence relation by Lemma \ref{union_connected}, and the partition of $X$ by $R$ is called the connected components of $X$.
\end{definition}
Let $C_x$ be the equivalence class containing $x$.
\begin{proposition}
    Connected components are nonempty and are maximal (wrt inclusion) connected subset of $X$, also they are closed.
\end{proposition}
\begin{proof}
    Let $C$ be a connected component, so it is the equivalence class of some $x$, so $C=C_x$, so $C$ is nonempty since it contains $X$.
    So given $y\in C$, $\exists A_y\ni x,y$ such that $U$ is connected.
    $A_y\in C$ by definition of the relation.
    Now $\forall y,z\in C, x\in A_y\cap A_z\neq\varnothing$, therefore by Lemma\ref{union_connected}, hence $C=\bigcup_{y\in C}A_y$ is connected.\\
    If $C\subset D$ and $D$ is connected, then $\forall y\in D$, $x,y\in D$, thus since $D$ is connected $y\in C$, so $D\subset C\implies C=D$.\\
    Hence since $\bar C$ is connected and contains $C$, by maximality $C=\bar C$, therefore $C$ is closed.
\end{proof}
\begin{definition}
    A topological space $X$ is called path-connected if $\forall x,y\in X,\exists\gamma:[0,1]\to X$ continuous, $\gamma(0)=x,\gamma(1)=y$.
\end{definition}
\begin{theorem}
    Any path-connected space is connected.
\end{theorem}
\begin{proof}
    Suppose not, then $X$ is path-connected but not connected, so there are open $U,V$ disconnects $X$.
    Then fixing $x\in U,y\in V$, there exists a continuous $\gamma:[0,1]\to X$ such that $\gamma(0)=x,\gamma(1)=y$.
    Thus $\gamma^{-1}(U),\gamma^{-1}(V)$ are nonempty, open, and partitions $[0,1]$, thus $[0,1]$ is disconnected by them, which is a contradiction.
\end{proof}
The converse, however, is not true.
\begin{example}
    Take the Topologist's Sine Wave, $X=\{(x,\sin(1/x)):x>0\}\cup(\{0\}\times [-1,1])$.
    We have already shown it is connected.
    But it is not path-connected.
    Indeed, pick points $(0,0),(1,\sin(1))\in X$.
    Assume that $\gamma:[0,1]\to X$ is continuous and $\gamma(0)=(0,0)=x,\gamma(1)=(1,\sin(1))=y$.
    Let $\gamma_1,\gamma_2$ be the components of $\gamma$, which are continuous.
    For $\gamma_1(t)>0$, then $[0,\gamma_1(t)]\subset \gamma_1([0,t])$ by IVT, so $\exists n\in\mathbb N,(2\pi n)^{-1},(2\pi n+\pi/2)^{-1}\in (0,\gamma_1(t))\subset \gamma_1([0,t])$.
    So there is some $a,b$ with $\gamma_1(a)=(2\pi n)^{-1},\gamma_1(b)=(2\pi n+\pi/2)^{-1}$, hence $\gamma_2(a)=0,\gamma_2(b)=1$, so we can thus find a sequence $1>t_1>t_2>\cdots>0$ with
    $$\gamma_2(t_n)=
    \begin{cases}
        1\text{, if $n$ is even}\\
        0\text{, otherwise}
    \end{cases}$$
    So $t_n$ converges but $\gamma_2(t_n)$ does not.
    This is a contradiction.
\end{example}
\begin{lemma}[Gluing Lemma]
    Let $f:X\to Y$ be a function between topological spaces.
    If $X=A\cup B$ where $A,B$ are closed and $f|_A,f|_B$ are continuous, then $f$ is continuous.
\end{lemma}
\begin{proof}
    Given closed $V$ in $Y$,
    $$f^{-1}(V)=(f^{-1}(V)\cap A)\cup (f^{-1}(V)\cap B)=(f|_A)^{-1}(V)\cup (f|_B)^{-1}(V)$$
    which is closed since $A,B$ are closed.
    Hence $f$ is continuous.
\end{proof}
\begin{corollary}
    Let $X$ be a topological space.
    Define the relation $R$ by $xRy$ if and only if there is a continuous $\gamma:[0,1]\to X$ such that $\gamma(0)=x,\gamma(1)=y$.
    Then this is an equivalence relation.
\end{corollary}
\begin{proof}
    Trivial.
\end{proof}
\begin{theorem}
    Let $U\subset\mathbb R^n$ be open, then $U$ is connected if and only if $U$ is path-connected.
\end{theorem}
\begin{proof}
    It suffice to show every open connected subset of $\mathbb R^n$ is path-connected.\\
    WLOG $U\neq\varnothing$, fix $x_0\in U$, let $V$ be the path-connected component containing $x_0$.
    We shall show that $V,U\setminus V$ are both open, so by assumption $V=U$, thus the proof will be done.\\
    $V$ open: Since $U$ is open, for any $x\in U$, there is $r>0$ such that $D_r(x)\in U$.
    But any ball is path connected, so $\forall x\in V,\exists r_x>0, D_{r_x}(x)\in V$, so $V$ is open.\\
    $U\setminus V$ open: Fix by the same proof as above, any path-connected components in $V$ is open, so since $U\setminus V$ is the union of some of them (the ones except $V$), it is open.
\end{proof}
\begin{example}
    For $n\ge 2$, $\mathbb R^n$ is not homeomorphic to $\mathbb R$.
    Assume $f:\mathbb R^n\to \mathbb R$ is a homeomorphism.
    Fix $x\in\mathbb R^n$, and let $y=f(x)$, then $f|_{\mathbb R^n\setminus\{x\}}$ is still a homeomorphism to $\mathbb R\setminus\{y\}$.
    But then $\mathbb R^n\setminus\{x\}$ is connected by the preceding theorem, but $\mathbb R\setminus\{y\}$ is not, contradiction.
\end{example}
